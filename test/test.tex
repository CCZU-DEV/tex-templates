\documentclass[a4paper, 11pt]{ctexart}

\usepackage{xfrac}
\usepackage{geometry}
\usepackage{fancyhdr}
\usepackage{lastpage}
\usepackage{minted}
\usepackage{fontspec}
\usepackage{graphicx}
\usepackage{adjustbox}
\usepackage{tabularx}
\usepackage{multirow}
\usepackage{dblfloatfix}
\usepackage{hyperref}
\usepackage{wrapfig}
\usepackage{float}
\usepackage{caption}
\usepackage{booktabs}
\usepackage{chngcntr}
\usepackage{amsmath}
\usepackage{amssymb}
\usepackage{ntheorem}
\usepackage[nottoc]{tocbibind}

% 设置页面
\geometry{hmargin=3.18cm, vmargin=2.54cm} % 页边距

% 添加空白页命令
\newcommand{\blankpage}{\phantom{s}\thispagestyle{empty}}

% 设置字体
\setmainfont{Noto Serif} % 如果没有安装 Noto 字体,可以把这三行注释掉,ctex 会使用默认 CJK 字体
\setsansfont{Noto Sans}
\setCJKsansfont{Noto Sans CJK SC}
\setmonofont[Scale=0.95]{Consolas}
\setCJKmonofont[Scale=0.95]{Microsoft YaHei}

% 设置代码默认显示效果
\setminted{autogobble, baselinestretch=1, breaklines, frame=lines, framesep=3mm, style=vs, stripnl, tabsize=4}
\floatname{listing}{代码清单}
\newcommand{\codeinline}{\mintinline{text}}

% 修正代码清单的标题上边距
\let\oldcaption\caption
\makeatletter
\renewcommand{\caption}[1]{
    \ifnum\strcmp{\@currenvir}{listing}=0
    \vspace{-0.3cm}\oldcaption{#1}
    \else
    \oldcaption{#1}
    \fi
}
\makeatother

% 设置表格
\newcolumntype{C}{>{\centering\arraybackslash}X}

% 设置标号
\counterwithin{figure}{section}
\counterwithin{table}{section}
\counterwithin{listing}{section}
\counterwithin{equation}{section}

% 设置定理、例题等特殊块
\theoremstyle{plain}
\theoremheaderfont{\kern+2em\normalfont\bfseries} % 首行缩进 2 格
\theorembodyfont{\normalfont\bfseries} % 定义、定理使用粗体
\newtheorem{definition}{定义}[section]
\newtheorem{theorem}{定理}[section]
\newtheorem{corollary}{推论}[theorem]
\newtheorem{lemma}[theorem]{引理}
\theorembodyfont{\normalfont} % 其它使用正常字体
\newtheorem*{proof}{证明}
\newtheorem{example}{例}[section]
\newtheorem*{solution}{解}
\newtheorem*{remark}{注}


\title{一篇 \LaTeX 的测试文章}
\author{某某某}

\begin{document}

\maketitle
\thispagestyle{empty}

\clearpage

\tableofcontents
\thispagestyle{empty}

\clearpage
\setcounter{page}{1}

\section{数学相关}

\subsection{基本}

数学相关的文章通常使用英文标点, \LaTeX 能正确处理中英文字和标点混杂的情况, 比如本 section 使用的是英文标点.

下面是个带标号的数学公式:

\begin{equation}
    \Gamma(\alpha) = \int_{0}^{+\infty} x^{\alpha - 1} e^{-x} \,dx (\alpha > 0)
    \label{eq:gamma-fn}
\end{equation}

公式 \ref{eq:gamma-fn} 是 $\Gamma$ 函数.

这一个行内公式: $f(x) = ax + b$, 以及另一个: $\int \cos x \,dx = \sin x$. 下面是没有标号的块公式:

$$
    E = mc^2
$$

\subsection{定理, 推论, 例题}

\begin{theorem}
    这里是一个很长很长很长很长很长很长很长很长很长很长很长很长很长很长很长很长很长很长很长很长很长很长定理.
    \label{th:1}
\end{theorem}

定理 \ref{th:1} 同样可以通过 \codeinline{\ref} 引用标号.

\begin{proof}
    这里是定理 \ref{th:1} 的证明.
\end{proof}

\begin{example}
    这是一个例题. 包含一个公式:

    \begin{equation}
        \Gamma(\alpha) = \int_{0}^{+\infty} x^{\alpha - 1} e^{-x} \,dx (\alpha > 0)
    \end{equation}

    \label{eg:1}
\end{example}

\begin{solution}
    这是例 \ref{eg:1} 的解.
\end{solution}

\begin{remark}
    这是一个注.
\end{remark}

\end{document}
